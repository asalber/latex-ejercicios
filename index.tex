% Options for packages loaded elsewhere
\PassOptionsToPackage{unicode}{hyperref}
\PassOptionsToPackage{hyphens}{url}
\PassOptionsToPackage{dvipsnames,svgnames,x11names}{xcolor}
%
\documentclass[
  a4paper,
]{scrreport}

\usepackage{amsmath,amssymb}
\usepackage{iftex}
\ifPDFTeX
  \usepackage[T1]{fontenc}
  \usepackage[utf8]{inputenc}
  \usepackage{textcomp} % provide euro and other symbols
\else % if luatex or xetex
  \usepackage{unicode-math}
  \defaultfontfeatures{Scale=MatchLowercase}
  \defaultfontfeatures[\rmfamily]{Ligatures=TeX,Scale=1}
\fi
\usepackage{lmodern}
\ifPDFTeX\else  
    % xetex/luatex font selection
\fi
% Use upquote if available, for straight quotes in verbatim environments
\IfFileExists{upquote.sty}{\usepackage{upquote}}{}
\IfFileExists{microtype.sty}{% use microtype if available
  \usepackage[]{microtype}
  \UseMicrotypeSet[protrusion]{basicmath} % disable protrusion for tt fonts
}{}
\makeatletter
\@ifundefined{KOMAClassName}{% if non-KOMA class
  \IfFileExists{parskip.sty}{%
    \usepackage{parskip}
  }{% else
    \setlength{\parindent}{0pt}
    \setlength{\parskip}{6pt plus 2pt minus 1pt}}
}{% if KOMA class
  \KOMAoptions{parskip=half}}
\makeatother
\usepackage{xcolor}
\setlength{\emergencystretch}{3em} % prevent overfull lines
\setcounter{secnumdepth}{5}
% Make \paragraph and \subparagraph free-standing
\ifx\paragraph\undefined\else
  \let\oldparagraph\paragraph
  \renewcommand{\paragraph}[1]{\oldparagraph{#1}\mbox{}}
\fi
\ifx\subparagraph\undefined\else
  \let\oldsubparagraph\subparagraph
  \renewcommand{\subparagraph}[1]{\oldsubparagraph{#1}\mbox{}}
\fi

\usepackage{color}
\usepackage{fancyvrb}
\newcommand{\VerbBar}{|}
\newcommand{\VERB}{\Verb[commandchars=\\\{\}]}
\DefineVerbatimEnvironment{Highlighting}{Verbatim}{commandchars=\\\{\}}
% Add ',fontsize=\small' for more characters per line
\usepackage{framed}
\definecolor{shadecolor}{RGB}{241,243,245}
\newenvironment{Shaded}{\begin{snugshade}}{\end{snugshade}}
\newcommand{\AlertTok}[1]{\textcolor[rgb]{0.68,0.00,0.00}{#1}}
\newcommand{\AnnotationTok}[1]{\textcolor[rgb]{0.37,0.37,0.37}{#1}}
\newcommand{\AttributeTok}[1]{\textcolor[rgb]{0.40,0.45,0.13}{#1}}
\newcommand{\BaseNTok}[1]{\textcolor[rgb]{0.68,0.00,0.00}{#1}}
\newcommand{\BuiltInTok}[1]{\textcolor[rgb]{0.00,0.23,0.31}{#1}}
\newcommand{\CharTok}[1]{\textcolor[rgb]{0.13,0.47,0.30}{#1}}
\newcommand{\CommentTok}[1]{\textcolor[rgb]{0.37,0.37,0.37}{#1}}
\newcommand{\CommentVarTok}[1]{\textcolor[rgb]{0.37,0.37,0.37}{\textit{#1}}}
\newcommand{\ConstantTok}[1]{\textcolor[rgb]{0.56,0.35,0.01}{#1}}
\newcommand{\ControlFlowTok}[1]{\textcolor[rgb]{0.00,0.23,0.31}{#1}}
\newcommand{\DataTypeTok}[1]{\textcolor[rgb]{0.68,0.00,0.00}{#1}}
\newcommand{\DecValTok}[1]{\textcolor[rgb]{0.68,0.00,0.00}{#1}}
\newcommand{\DocumentationTok}[1]{\textcolor[rgb]{0.37,0.37,0.37}{\textit{#1}}}
\newcommand{\ErrorTok}[1]{\textcolor[rgb]{0.68,0.00,0.00}{#1}}
\newcommand{\ExtensionTok}[1]{\textcolor[rgb]{0.00,0.23,0.31}{#1}}
\newcommand{\FloatTok}[1]{\textcolor[rgb]{0.68,0.00,0.00}{#1}}
\newcommand{\FunctionTok}[1]{\textcolor[rgb]{0.28,0.35,0.67}{#1}}
\newcommand{\ImportTok}[1]{\textcolor[rgb]{0.00,0.46,0.62}{#1}}
\newcommand{\InformationTok}[1]{\textcolor[rgb]{0.37,0.37,0.37}{#1}}
\newcommand{\KeywordTok}[1]{\textcolor[rgb]{0.00,0.23,0.31}{#1}}
\newcommand{\NormalTok}[1]{\textcolor[rgb]{0.00,0.23,0.31}{#1}}
\newcommand{\OperatorTok}[1]{\textcolor[rgb]{0.37,0.37,0.37}{#1}}
\newcommand{\OtherTok}[1]{\textcolor[rgb]{0.00,0.23,0.31}{#1}}
\newcommand{\PreprocessorTok}[1]{\textcolor[rgb]{0.68,0.00,0.00}{#1}}
\newcommand{\RegionMarkerTok}[1]{\textcolor[rgb]{0.00,0.23,0.31}{#1}}
\newcommand{\SpecialCharTok}[1]{\textcolor[rgb]{0.37,0.37,0.37}{#1}}
\newcommand{\SpecialStringTok}[1]{\textcolor[rgb]{0.13,0.47,0.30}{#1}}
\newcommand{\StringTok}[1]{\textcolor[rgb]{0.13,0.47,0.30}{#1}}
\newcommand{\VariableTok}[1]{\textcolor[rgb]{0.07,0.07,0.07}{#1}}
\newcommand{\VerbatimStringTok}[1]{\textcolor[rgb]{0.13,0.47,0.30}{#1}}
\newcommand{\WarningTok}[1]{\textcolor[rgb]{0.37,0.37,0.37}{\textit{#1}}}

\providecommand{\tightlist}{%
  \setlength{\itemsep}{0pt}\setlength{\parskip}{0pt}}\usepackage{longtable,booktabs,array}
\usepackage{calc} % for calculating minipage widths
% Correct order of tables after \paragraph or \subparagraph
\usepackage{etoolbox}
\makeatletter
\patchcmd\longtable{\par}{\if@noskipsec\mbox{}\fi\par}{}{}
\makeatother
% Allow footnotes in longtable head/foot
\IfFileExists{footnotehyper.sty}{\usepackage{footnotehyper}}{\usepackage{footnote}}
\makesavenoteenv{longtable}
\usepackage{graphicx}
\makeatletter
\def\maxwidth{\ifdim\Gin@nat@width>\linewidth\linewidth\else\Gin@nat@width\fi}
\def\maxheight{\ifdim\Gin@nat@height>\textheight\textheight\else\Gin@nat@height\fi}
\makeatother
% Scale images if necessary, so that they will not overflow the page
% margins by default, and it is still possible to overwrite the defaults
% using explicit options in \includegraphics[width, height, ...]{}
\setkeys{Gin}{width=\maxwidth,height=\maxheight,keepaspectratio}
% Set default figure placement to htbp
\makeatletter
\def\fps@figure{htbp}
\makeatother

%\newfontfamily\Ubuntu[Mapping=tex-text]{Ubuntu}
\usepackage{pgfplots}
\usetikzlibrary{arrows.meta,arrows}
\usetikzlibrary{angles,quotes}
\pgfplotsset{grid style={dashed,mygray}}
% Colors
\definecolor{myblue}{rgb}{0.067,0.529,0.871}
\definecolor{mypurple}{rgb}{0.859,0.071,0.525}
\definecolor{myred}{rgb}{1.0, 0.13, 0.32}
\definecolor{mygreen}{rgb}{0.01, 0.75, 0.24}
\definecolor{myblack}{gray}{0.1}
\definecolor{mygray}{gray}{0.8}
\newcommand{\NN}{\mathbb{N}}
\newcommand{\ZZ}{\mathbb{Z}}
\newcommand{\QQ}{\mathbb{Q}}
\newcommand{\RR}{\mathbb{R}}
\newcommand{\CC}{\mathbb{C}}
\DeclareMathOperator{\Int}{Int}
\DeclareMathOperator{\Ext}{Ext}
\DeclareMathOperator{\Fr}{Fr}
\DeclareMathOperator{\Adh}{Adh}
\DeclareMathOperator{\Ac}{Ac}
\DeclareMathOperator{\sen}{sen}
\makeatletter
\@ifpackageloaded{tcolorbox}{}{\usepackage[skins,breakable]{tcolorbox}}
\@ifpackageloaded{fontawesome5}{}{\usepackage{fontawesome5}}
\definecolor{quarto-callout-color}{HTML}{909090}
\definecolor{quarto-callout-note-color}{HTML}{0758E5}
\definecolor{quarto-callout-important-color}{HTML}{CC1914}
\definecolor{quarto-callout-warning-color}{HTML}{EB9113}
\definecolor{quarto-callout-tip-color}{HTML}{00A047}
\definecolor{quarto-callout-caution-color}{HTML}{FC5300}
\definecolor{quarto-callout-color-frame}{HTML}{acacac}
\definecolor{quarto-callout-note-color-frame}{HTML}{4582ec}
\definecolor{quarto-callout-important-color-frame}{HTML}{d9534f}
\definecolor{quarto-callout-warning-color-frame}{HTML}{f0ad4e}
\definecolor{quarto-callout-tip-color-frame}{HTML}{02b875}
\definecolor{quarto-callout-caution-color-frame}{HTML}{fd7e14}
\makeatother
\makeatletter
\makeatother
\makeatletter
\@ifpackageloaded{bookmark}{}{\usepackage{bookmark}}
\makeatother
\makeatletter
\@ifpackageloaded{caption}{}{\usepackage{caption}}
\AtBeginDocument{%
\ifdefined\contentsname
  \renewcommand*\contentsname{Tabla de contenidos}
\else
  \newcommand\contentsname{Tabla de contenidos}
\fi
\ifdefined\listfigurename
  \renewcommand*\listfigurename{Listado de Figuras}
\else
  \newcommand\listfigurename{Listado de Figuras}
\fi
\ifdefined\listtablename
  \renewcommand*\listtablename{Listado de Tablas}
\else
  \newcommand\listtablename{Listado de Tablas}
\fi
\ifdefined\figurename
  \renewcommand*\figurename{Figura}
\else
  \newcommand\figurename{Figura}
\fi
\ifdefined\tablename
  \renewcommand*\tablename{Tabla}
\else
  \newcommand\tablename{Tabla}
\fi
}
\@ifpackageloaded{float}{}{\usepackage{float}}
\floatstyle{ruled}
\@ifundefined{c@chapter}{\newfloat{codelisting}{h}{lop}}{\newfloat{codelisting}{h}{lop}[chapter]}
\floatname{codelisting}{Listado}
\newcommand*\listoflistings{\listof{codelisting}{Listado de Listados}}
\usepackage{amsthm}
\theoremstyle{definition}
\newtheorem{exercise}{Ejercicio}[chapter]
\theoremstyle{remark}
\AtBeginDocument{\renewcommand*{\proofname}{Prueba}}
\newtheorem*{remark}{Observación}
\newtheorem*{solution}{Solución}
\makeatother
\makeatletter
\@ifpackageloaded{caption}{}{\usepackage{caption}}
\@ifpackageloaded{subcaption}{}{\usepackage{subcaption}}
\makeatother
\makeatletter
\@ifpackageloaded{tcolorbox}{}{\usepackage[skins,breakable]{tcolorbox}}
\makeatother
\makeatletter
\@ifundefined{shadecolor}{\definecolor{shadecolor}{rgb}{.97, .97, .97}}
\makeatother
\makeatletter
\makeatother
\makeatletter
\makeatother
\ifLuaTeX
\usepackage[bidi=basic]{babel}
\else
\usepackage[bidi=default]{babel}
\fi
\babelprovide[main,import]{spanish}
% get rid of language-specific shorthands (see #6817):
\let\LanguageShortHands\languageshorthands
\def\languageshorthands#1{}
\ifLuaTeX
  \usepackage{selnolig}  % disable illegal ligatures
\fi
\IfFileExists{bookmark.sty}{\usepackage{bookmark}}{\usepackage{hyperref}}
\IfFileExists{xurl.sty}{\usepackage{xurl}}{} % add URL line breaks if available
\urlstyle{same} % disable monospaced font for URLs
\hypersetup{
  pdftitle={Ejercicios de LaTeX},
  pdfauthor={Alfredo Sánchez Alberca},
  pdflang={es},
  colorlinks=true,
  linkcolor={blue},
  filecolor={Maroon},
  citecolor={Blue},
  urlcolor={Blue},
  pdfcreator={LaTeX via pandoc}}

\title{Ejercicios de LaTeX}
\author{Alfredo Sánchez Alberca}
\date{2023-01-06}

\begin{document}
\maketitle
\ifdefined\Shaded\renewenvironment{Shaded}{\begin{tcolorbox}[interior hidden, breakable, enhanced, sharp corners, boxrule=0pt, frame hidden, borderline west={3pt}{0pt}{shadecolor}]}{\end{tcolorbox}}\fi

\renewcommand*\contentsname{Tabla de contenidos}
{
\hypersetup{linkcolor=}
\setcounter{tocdepth}{2}
\tableofcontents
}
\bookmarksetup{startatroot}

\hypertarget{prefacio}{%
\chapter*{Prefacio}\label{prefacio}}
\addcontentsline{toc}{chapter}{Prefacio}

\markboth{Prefacio}{Prefacio}

¡Bienvenida/os a Ejercicios de \LaTeX!

En esta página puedes encontrar una colección de ejercicios para
practicar con el lenguaje de composición de textos \LaTeX en la
creación de documentos científicos y técnicos.

\hypertarget{licencia}{%
\section*{Licencia}\label{licencia}}
\addcontentsline{toc}{section}{Licencia}

\markright{Licencia}

Esta obra está bajo una licencia Reconocimiento -- No comercial --
Compartir bajo la misma licencia 3.0 España de Creative Commons. Para
ver una copia de esta licencia, visite
\url{https://creativecommons.org/licenses/by-nc-sa/3.0/es/}.

Con esta licencia eres libre de:

\begin{itemize}
\tightlist
\item
  Copiar, distribuir y mostrar este trabajo.
\item
  Realizar modificaciones de este trabajo.
\end{itemize}

Bajo las siguientes condiciones:

\begin{itemize}
\item
  \textbf{Reconocimiento}. Debe reconocer los créditos de la obra de la
  manera especificada por el autor o el licenciador (pero no de una
  manera que sugiera que tiene su apoyo o apoyan el uso que hace de su
  obra).
\item
  \textbf{No comercial}. No puede utilizar esta obra para fines
  comerciales.
\item
  \textbf{Compartir bajo la misma licencia}. Si altera o transforma esta
  obra, o genera una obra derivada, sólo puede distribuir la obra
  generada bajo una licencia idéntica a ésta.
\end{itemize}

Al reutilizar o distribuir la obra, tiene que dejar bien claro los
términos de la licencia de esta obra.

Estas condiciones pueden no aplicarse si se obtiene el permiso del
titular de los derechos de autor.

Nada en esta licencia menoscaba o restringe los derechos morales del
autor.

\bookmarksetup{startatroot}

\hypertarget{ejercicios-buxe1sicos}{%
\chapter{Ejercicios básicos}\label{ejercicios-buxe1sicos}}

\begin{exercise}[]\protect\hypertarget{exr-hola-mundo}{}\label{exr-hola-mundo}

Escribir el código LaTeX para generar el siguiente
\href{doc/ejercicio1.pdf}{documento}, pero con la fecha actual.

\begin{tcolorbox}[enhanced jigsaw, left=2mm, colbacktitle=quarto-callout-note-color!10!white, toprule=.15mm, coltitle=black, leftrule=.75mm, arc=.35mm, colback=white, opacitybacktitle=0.6, rightrule=.15mm, colframe=quarto-callout-note-color-frame, title=\textcolor{quarto-callout-note-color}{\faInfo}\hspace{0.5em}{Ver documento}, bottomrule=.15mm, breakable, opacityback=0, bottomtitle=1mm, toptitle=1mm, titlerule=0mm]

\end{tcolorbox}

\end{exercise}

\begin{tcolorbox}[enhanced jigsaw, left=2mm, colbacktitle=quarto-callout-tip-color!10!white, toprule=.15mm, coltitle=black, leftrule=.75mm, arc=.35mm, colback=white, opacitybacktitle=0.6, rightrule=.15mm, colframe=quarto-callout-tip-color-frame, title=\textcolor{quarto-callout-tip-color}{\faLightbulb}\hspace{0.5em}{Solución}, bottomrule=.15mm, breakable, opacityback=0, bottomtitle=1mm, toptitle=1mm, titlerule=0mm]

\begin{Shaded}
\begin{Highlighting}[]
\CommentTok{\% CLASE}
\BuiltInTok{\textbackslash{}documentclass}\NormalTok{[a4paper,12pt]\{}\ExtensionTok{article}\NormalTok{\}}
\CommentTok{\% PREÁMBULO}
\CommentTok{\% Paquetes}
\BuiltInTok{\textbackslash{}usepackage}\NormalTok{\{}\ExtensionTok{fontspec}\NormalTok{\}}
\BuiltInTok{\textbackslash{}usepackage}\NormalTok{\{}\ExtensionTok{polyglossia}\NormalTok{\}}
\FunctionTok{\textbackslash{}setdefaultlanguage}\NormalTok{\{spanish\}}

\CommentTok{\% Título, autor y fecha}
\FunctionTok{\textbackslash{}title}\NormalTok{\{Hola Mundo\}}
\FunctionTok{\textbackslash{}author}\NormalTok{\{Alfredo Sánchez Alberca\}}
\FunctionTok{\textbackslash{}date}\NormalTok{\{}\FunctionTok{\textbackslash{}today}\NormalTok{\}}

\KeywordTok{\textbackslash{}begin}\NormalTok{\{}\ExtensionTok{document}\NormalTok{\}}
\FunctionTok{\textbackslash{}maketitle}

\FunctionTok{\textbackslash{}textbf}\NormalTok{\{¡Hola Mundo!\} Hoy empiezo a aprender }\FunctionTok{\textbackslash{}LaTeX}\NormalTok{.}
\KeywordTok{\textbackslash{}end}\NormalTok{\{}\ExtensionTok{document}\NormalTok{\}}
\end{Highlighting}
\end{Shaded}

\end{tcolorbox}

\begin{exercise}[]\protect\hypertarget{exr-articulo-breve}{}\label{exr-articulo-breve}

Escribir el código LaTeX para generar el siguiente
\href{doc/ejercicio2.pdf}{documento}.

\begin{tcolorbox}[enhanced jigsaw, left=2mm, colbacktitle=quarto-callout-note-color!10!white, toprule=.15mm, coltitle=black, leftrule=.75mm, arc=.35mm, colback=white, opacitybacktitle=0.6, rightrule=.15mm, colframe=quarto-callout-note-color-frame, title=\textcolor{quarto-callout-note-color}{\faInfo}\hspace{0.5em}{Ver documento}, bottomrule=.15mm, breakable, opacityback=0, bottomtitle=1mm, toptitle=1mm, titlerule=0mm]

\end{tcolorbox}

\end{exercise}

\begin{tcolorbox}[enhanced jigsaw, left=2mm, colbacktitle=quarto-callout-tip-color!10!white, toprule=.15mm, coltitle=black, leftrule=.75mm, arc=.35mm, colback=white, opacitybacktitle=0.6, rightrule=.15mm, colframe=quarto-callout-tip-color-frame, title=\textcolor{quarto-callout-tip-color}{\faLightbulb}\hspace{0.5em}{Solución}, bottomrule=.15mm, breakable, opacityback=0, bottomtitle=1mm, toptitle=1mm, titlerule=0mm]

\begin{Shaded}
\begin{Highlighting}[]
\CommentTok{\% CLASE}
\BuiltInTok{\textbackslash{}documentclass}\NormalTok{[a4paper,12pt]\{}\ExtensionTok{article}\NormalTok{\}}

\CommentTok{\% PREÁMBULO}
\CommentTok{\% Paquetes}
\BuiltInTok{\textbackslash{}usepackage}\NormalTok{\{}\ExtensionTok{fontspec}\NormalTok{\}}
\BuiltInTok{\textbackslash{}usepackage}\NormalTok{\{}\ExtensionTok{polyglossia}\NormalTok{\}}
\FunctionTok{\textbackslash{}setdefaultlanguage}\NormalTok{\{spanish\}}

\CommentTok{\% Título, autor y fecha}
\FunctionTok{\textbackslash{}title}\NormalTok{\{Curso de }\FunctionTok{\textbackslash{}LaTeX}\NormalTok{\}}
\FunctionTok{\textbackslash{}author}\NormalTok{\{María López }\FunctionTok{\textbackslash{}and}\NormalTok{ Juan Sánchez\}}
\FunctionTok{\textbackslash{}date}\NormalTok{\{\}}

\CommentTok{\% CUERPO}
\KeywordTok{\textbackslash{}begin}\NormalTok{\{}\ExtensionTok{document}\NormalTok{\}}
\FunctionTok{\textbackslash{}maketitle}
\FunctionTok{\textbackslash{}tableofcontents}

\KeywordTok{\textbackslash{}section}\NormalTok{\{Introducción\}}

\NormalTok{LaTeX es un sistema de composición de textos, orientado especialmente a la creación de documentos científicos y técnicos que contengan fórmulas matemáticas.}

\KeywordTok{\textbackslash{}subsection}\NormalTok{\{Código abierto\}}

\NormalTok{LaTeX es un programa de }\FunctionTok{\textbackslash{}emph}\NormalTok{\{código abierto\} por lo que cualquier usuario puede modificar el código y adaptarlo a sus necesidades.}

\KeywordTok{\textbackslash{}subsection}\NormalTok{\{Distribuciones\}}

\NormalTok{Existen distribuciones de LaTeX para la mayoría de los sistemas operativos. Las más conocidas son TexLive, MikTex y MacTex.}

\KeywordTok{\textbackslash{}section}\NormalTok{\{Paquetes\}}

\NormalTok{Existen multitud de paquetes de macros en LaTeX para realizar diversas tareas desde gráficos hasta composición de partituras. El principal repositorio de paquetes es CRAN( }\FunctionTok{\textbackslash{}texttt}\NormalTok{\{https://cran.r{-}project.org/\})}

\KeywordTok{\textbackslash{}end}\NormalTok{\{}\ExtensionTok{document}\NormalTok{\}}
\end{Highlighting}
\end{Shaded}

\end{tcolorbox}

\begin{exercise}[]\protect\hypertarget{exr-listas-anidadas}{}\label{exr-listas-anidadas}

Escribir el código LaTeX para generar el siguiente
\href{doc/ejercicio3.pdf}{documento}.

\begin{tcolorbox}[enhanced jigsaw, left=2mm, colbacktitle=quarto-callout-note-color!10!white, toprule=.15mm, coltitle=black, leftrule=.75mm, arc=.35mm, colback=white, opacitybacktitle=0.6, rightrule=.15mm, colframe=quarto-callout-note-color-frame, title=\textcolor{quarto-callout-note-color}{\faInfo}\hspace{0.5em}{Ver documento}, bottomrule=.15mm, breakable, opacityback=0, bottomtitle=1mm, toptitle=1mm, titlerule=0mm]

\end{tcolorbox}

\end{exercise}

\begin{tcolorbox}[enhanced jigsaw, left=2mm, colbacktitle=quarto-callout-tip-color!10!white, toprule=.15mm, coltitle=black, leftrule=.75mm, arc=.35mm, colback=white, opacitybacktitle=0.6, rightrule=.15mm, colframe=quarto-callout-tip-color-frame, title=\textcolor{quarto-callout-tip-color}{\faLightbulb}\hspace{0.5em}{Solución}, bottomrule=.15mm, breakable, opacityback=0, bottomtitle=1mm, toptitle=1mm, titlerule=0mm]

\begin{Shaded}
\begin{Highlighting}[]
\CommentTok{\% CLASE}
\BuiltInTok{\textbackslash{}documentclass}\NormalTok{\{}\ExtensionTok{article}\NormalTok{\}}

\CommentTok{\% PREAMBULO}
\BuiltInTok{\textbackslash{}usepackage}\NormalTok{[spanish]\{}\ExtensionTok{babel}\NormalTok{\}}

\CommentTok{\% CUERPO}
\KeywordTok{\textbackslash{}begin}\NormalTok{\{}\ExtensionTok{document}\NormalTok{\}}
\KeywordTok{\textbackslash{}section}\NormalTok{\{Contenidos del curso de }\FunctionTok{\textbackslash{}LaTeX}\NormalTok{\}}

\KeywordTok{\textbackslash{}begin}\NormalTok{\{}\ExtensionTok{enumerate}\NormalTok{\}}
  \FunctionTok{\textbackslash{}item}\NormalTok{ Instalación de }\FunctionTok{\textbackslash{}LaTeX}\NormalTok{.}
  \KeywordTok{\textbackslash{}begin}\NormalTok{\{}\ExtensionTok{enumerate}\NormalTok{\}}
    \FunctionTok{\textbackslash{}item}\NormalTok{ Versión para Windows.}
    \FunctionTok{\textbackslash{}item}\NormalTok{ Versión para Mac.}
    \FunctionTok{\textbackslash{}item}\NormalTok{ Versión para Linux.}
  \KeywordTok{\textbackslash{}end}\NormalTok{\{}\ExtensionTok{enumerate}\NormalTok{\}}

  \FunctionTok{\textbackslash{}item}\NormalTok{ Estructura de un documento.}
  \KeywordTok{\textbackslash{}begin}\NormalTok{\{}\ExtensionTok{enumerate}\NormalTok{\}}
    \FunctionTok{\textbackslash{}item}\NormalTok{ Capítulos y secciones.}
    \FunctionTok{\textbackslash{}item}\NormalTok{ Formateo de texto.}
  \KeywordTok{\textbackslash{}end}\NormalTok{\{}\ExtensionTok{enumerate}\NormalTok{\}}
  
  \FunctionTok{\textbackslash{}item}\NormalTok{ Listas.}
  \KeywordTok{\textbackslash{}begin}\NormalTok{\{}\ExtensionTok{enumerate}\NormalTok{\}}
    \FunctionTok{\textbackslash{}item}\NormalTok{ Enumeradas.}
    \FunctionTok{\textbackslash{}item}\NormalTok{ No numeradas.}
    \FunctionTok{\textbackslash{}item}\NormalTok{ Descriptivas.}
  \KeywordTok{\textbackslash{}end}\NormalTok{\{}\ExtensionTok{enumerate}\NormalTok{\}}
  
  \FunctionTok{\textbackslash{}item}\NormalTok{ Fórmulas matemáticas.}
  \KeywordTok{\textbackslash{}begin}\NormalTok{\{}\ExtensionTok{enumerate}\NormalTok{\}}
    \FunctionTok{\textbackslash{}item}\NormalTok{ Subíndices y superíndices.}
    \FunctionTok{\textbackslash{}item}\NormalTok{ Operadores matemáticos.}
    \FunctionTok{\textbackslash{}item}\NormalTok{ Vectores y matrices.}
  \KeywordTok{\textbackslash{}end}\NormalTok{\{}\ExtensionTok{enumerate}\NormalTok{\}}
\KeywordTok{\textbackslash{}end}\NormalTok{\{}\ExtensionTok{enumerate}\NormalTok{\}}
\KeywordTok{\textbackslash{}end}\NormalTok{\{}\ExtensionTok{document}\NormalTok{\}}
\end{Highlighting}
\end{Shaded}

\end{tcolorbox}

\begin{exercise}[]\protect\hypertarget{exr-tablas-horario}{}\label{exr-tablas-horario}

Escribir el código LaTeX para generar el siguiente
\href{doc/ejercicio4.pdf}{documento}.

\begin{tcolorbox}[enhanced jigsaw, left=2mm, colbacktitle=quarto-callout-note-color!10!white, toprule=.15mm, coltitle=black, leftrule=.75mm, arc=.35mm, colback=white, opacitybacktitle=0.6, rightrule=.15mm, colframe=quarto-callout-note-color-frame, title=\textcolor{quarto-callout-note-color}{\faInfo}\hspace{0.5em}{Ver documento}, bottomrule=.15mm, breakable, opacityback=0, bottomtitle=1mm, toptitle=1mm, titlerule=0mm]

\end{tcolorbox}

\end{exercise}

\begin{tcolorbox}[enhanced jigsaw, left=2mm, colbacktitle=quarto-callout-tip-color!10!white, toprule=.15mm, coltitle=black, leftrule=.75mm, arc=.35mm, colback=white, opacitybacktitle=0.6, rightrule=.15mm, colframe=quarto-callout-tip-color-frame, title=\textcolor{quarto-callout-tip-color}{\faLightbulb}\hspace{0.5em}{Solución}, bottomrule=.15mm, breakable, opacityback=0, bottomtitle=1mm, toptitle=1mm, titlerule=0mm]

\begin{Shaded}
\begin{Highlighting}[]
\CommentTok{\% CLASE}
\BuiltInTok{\textbackslash{}documentclass}\NormalTok{\{}\ExtensionTok{article}\NormalTok{\}}

\CommentTok{\% PREAMBULO}
\BuiltInTok{\textbackslash{}usepackage}\NormalTok{[spanish]\{}\ExtensionTok{babel}\NormalTok{\}}

\CommentTok{\% CUERPO}
\KeywordTok{\textbackslash{}begin}\NormalTok{\{}\ExtensionTok{document}\NormalTok{\}}

\KeywordTok{\textbackslash{}section*}\NormalTok{\{Horarios\}}

\KeywordTok{\textbackslash{}begin}\NormalTok{\{}\ExtensionTok{center}\NormalTok{\}}
\KeywordTok{\textbackslash{}begin}\NormalTok{\{}\ExtensionTok{tabular}\NormalTok{\}\{|l|c|c|c|c|c|\}}
\FunctionTok{\textbackslash{}hline}
 \OperatorTok{\&}\NormalTok{ lunes }\OperatorTok{\&}\NormalTok{ martes }\OperatorTok{\&}\NormalTok{ miércoles }\OperatorTok{\&}\NormalTok{ jueves }\OperatorTok{\&}\NormalTok{ viernes }\FunctionTok{\textbackslash{}\textbackslash{}}
\FunctionTok{\textbackslash{}hline}
\NormalTok{9:30{-}10:30 }\OperatorTok{\&}\NormalTok{ Matemáticas }\OperatorTok{\&}\NormalTok{ Matemáticas }\OperatorTok{\&}\NormalTok{ Física }\OperatorTok{\&}\NormalTok{ Matemáticas }\OperatorTok{\&}\NormalTok{ Física }\FunctionTok{\textbackslash{}\textbackslash{}}
\FunctionTok{\textbackslash{}hline}
\NormalTok{10:30{-}11:30 }\OperatorTok{\&}\NormalTok{ Química }\OperatorTok{\&}\NormalTok{ Física }\OperatorTok{\&}\NormalTok{ Química }\OperatorTok{\&}\NormalTok{ Física }\OperatorTok{\&}\NormalTok{ Química }\FunctionTok{\textbackslash{}\textbackslash{}}
\FunctionTok{\textbackslash{}hline}
\NormalTok{11:30{-}12:30 }\OperatorTok{\&}\NormalTok{ Física }\OperatorTok{\&} \OperatorTok{\&}\NormalTok{ Matemáticas }\OperatorTok{\&}\NormalTok{ Química }\OperatorTok{\&}\NormalTok{ Matemáticas }\FunctionTok{\textbackslash{}\textbackslash{}}
\FunctionTok{\textbackslash{}hline}
\KeywordTok{\textbackslash{}end}\NormalTok{\{}\ExtensionTok{tabular}\NormalTok{\}}
\KeywordTok{\textbackslash{}end}\NormalTok{\{}\ExtensionTok{center}\NormalTok{\}}

\KeywordTok{\textbackslash{}end}\NormalTok{\{}\ExtensionTok{document}\NormalTok{\}}
\end{Highlighting}
\end{Shaded}

\end{tcolorbox}

\begin{exercise}[]\protect\hypertarget{exr-imagenes-basico}{}\label{exr-imagenes-basico}

Escribir el código LaTeX para generar el siguiente
\href{doc/ejercicio4.pdf}{documento}. Las imágenes pueden descargarse
desde la Wikipedia.

\begin{tcolorbox}[enhanced jigsaw, left=2mm, colbacktitle=quarto-callout-note-color!10!white, toprule=.15mm, coltitle=black, leftrule=.75mm, arc=.35mm, colback=white, opacitybacktitle=0.6, rightrule=.15mm, colframe=quarto-callout-note-color-frame, title=\textcolor{quarto-callout-note-color}{\faInfo}\hspace{0.5em}{Ver documento}, bottomrule=.15mm, breakable, opacityback=0, bottomtitle=1mm, toptitle=1mm, titlerule=0mm]

\end{tcolorbox}

\end{exercise}

\begin{tcolorbox}[enhanced jigsaw, left=2mm, colbacktitle=quarto-callout-tip-color!10!white, toprule=.15mm, coltitle=black, leftrule=.75mm, arc=.35mm, colback=white, opacitybacktitle=0.6, rightrule=.15mm, colframe=quarto-callout-tip-color-frame, title=\textcolor{quarto-callout-tip-color}{\faLightbulb}\hspace{0.5em}{Solución}, bottomrule=.15mm, breakable, opacityback=0, bottomtitle=1mm, toptitle=1mm, titlerule=0mm]

\begin{Shaded}
\begin{Highlighting}[]
\CommentTok{\% CLASE}
\BuiltInTok{\textbackslash{}documentclass}\NormalTok{\{}\ExtensionTok{article}\NormalTok{\}}

\CommentTok{\% PREAMBULO}
\BuiltInTok{\textbackslash{}usepackage}\NormalTok{[spanish]\{}\ExtensionTok{babel}\NormalTok{\}}
\BuiltInTok{\textbackslash{}usepackage}\NormalTok{\{}\ExtensionTok{graphicx}\NormalTok{\}}

\CommentTok{\% CUERPO}
\KeywordTok{\textbackslash{}begin}\NormalTok{\{}\ExtensionTok{document}\NormalTok{\}}

\NormalTok{El creador de }\FunctionTok{\textbackslash{}TeX\textbackslash{} }\NormalTok{fue Donald Knutt y el de }\FunctionTok{\textbackslash{}LaTeX\textbackslash{} }\NormalTok{Leslie Lamport.}

\KeywordTok{\textbackslash{}begin}\NormalTok{\{}\ExtensionTok{figure}\NormalTok{\}[!h]}
  \FunctionTok{\textbackslash{}centering}
  \BuiltInTok{\textbackslash{}includegraphics}\NormalTok{[height=7cm]\{}\ExtensionTok{img/Donald\_Knuth.jpg}\NormalTok{\}}
  \FunctionTok{\textbackslash{}caption}\NormalTok{\{Donald Knutt\}}
  \KeywordTok{\textbackslash{}label}\NormalTok{\{}\ExtensionTok{fig:knutt}\NormalTok{\}}
\KeywordTok{\textbackslash{}end}\NormalTok{\{}\ExtensionTok{figure}\NormalTok{\}}

\KeywordTok{\textbackslash{}begin}\NormalTok{\{}\ExtensionTok{figure}\NormalTok{\}[!h]}
  \FunctionTok{\textbackslash{}centering}
  \BuiltInTok{\textbackslash{}includegraphics}\NormalTok{[height=7cm]\{}\ExtensionTok{img/Leslie\_Lamport.jpg}\NormalTok{\}}
  \FunctionTok{\textbackslash{}caption}\NormalTok{\{Leslie Lamport\}}
  \KeywordTok{\textbackslash{}label}\NormalTok{\{}\ExtensionTok{fig:knutt}\NormalTok{\}}
  \KeywordTok{\textbackslash{}end}\NormalTok{\{}\ExtensionTok{figure}\NormalTok{\}}

\KeywordTok{\textbackslash{}end}\NormalTok{\{}\ExtensionTok{document}\NormalTok{\}}
\end{Highlighting}
\end{Shaded}

\end{tcolorbox}

\begin{exercise}[]\protect\hypertarget{exr-formulas-basicas}{}\label{exr-formulas-basicas}

Escribir el código LaTeX para generar las siguientes fórmulas:

\begin{enumerate}
\def\labelenumi{\alph{enumi}.}
\item
  \[\int_a^b x dx = \left.\frac{x^2}{2} \right|_a^b\]
\item
  \[\frac{dy}{dx}=y'=\lim_{h \to 0}\frac{f(x+h)-f(x)}{h}\]
\item
  \[\sum_n \frac{1}{n^s}=\prod_p \frac{1}{1-\frac{1}{p^s}}\]
\item
  \[\nabla f(x,y,z) = \left(\frac{\partial f}{\partial x}, \frac{\partial f}{\partial y}, \frac{\partial f}{\partial z}\right)\]
\item
  \[\frac{1+\frac{a}{b}}{1+\ln{\frac{\sqrt{b^2}}{1+\frac{1}{a}}}}\]
\end{enumerate}

\end{exercise}

\begin{tcolorbox}[enhanced jigsaw, left=2mm, colbacktitle=quarto-callout-tip-color!10!white, toprule=.15mm, coltitle=black, leftrule=.75mm, arc=.35mm, colback=white, opacitybacktitle=0.6, rightrule=.15mm, colframe=quarto-callout-tip-color-frame, title=\textcolor{quarto-callout-tip-color}{\faLightbulb}\hspace{0.5em}{Solución}, bottomrule=.15mm, breakable, opacityback=0, bottomtitle=1mm, toptitle=1mm, titlerule=0mm]

\begin{enumerate}
\def\labelenumi{\alph{enumi}.}
\item
  \texttt{\$\$\textbackslash{}int\_a\^{}b\ x\textbackslash{},dx\ =\ \textbackslash{}left.\textbackslash{}frac\{x\^{}2\}\{2\}\ \textbackslash{}right\textbar{}\_a\^{}b\$\$}
\item
  \texttt{\$\$\textbackslash{}frac\{dy\}\{dx\}=y\textquotesingle{}=\textbackslash{}lim\_\{h\ \textbackslash{}to\ 0\}\textbackslash{}frac\{f(x+h)-f(x)\}\{h\}\$\$}
\item
  \texttt{\textbackslash{}sum\_n\ \textbackslash{}frac\{1\}\{n\^{}s\}=\textbackslash{}prod\_p\ \textbackslash{}frac\{1\}\{1-\textbackslash{}frac\{1\}\{p\^{}s\}\}}
\item
  \texttt{\$\$\textbackslash{}nabla\ f(x,y,z)\ =\ \textbackslash{}left(\textbackslash{}frac\{\textbackslash{}partial\ f\}\{\textbackslash{}partial\ x\},\ \textbackslash{}frac\{\textbackslash{}partial\ f\}\{\textbackslash{}partial\ y\},\ \textbackslash{}frac\{\textbackslash{}partial\ f\}\{\textbackslash{}partial\ z\}\textbackslash{}right)\$\$}
\item
  \texttt{\$\$\textbackslash{}frac\{1+\textbackslash{}frac\{a\}\{b\}\}\{1+\textbackslash{}ln\{\textbackslash{}frac\{\textbackslash{}sqrt\{b\^{}2\}\}\{1+\textbackslash{}frac\{1\}\{a\}\}\}\}\$\$}
\end{enumerate}

\end{tcolorbox}

\begin{exercise}[]\protect\hypertarget{exr-matrices-basicas}{}\label{exr-matrices-basicas}

Escribir el código LaTeX para generar la siguiente expresión matricial:

\[
\begin{bmatrix}
a_{11} & a_{12} & \dots & a_{1m} \\
a_{21} & a_{22} & \dots & a_{2m} \\
\vdots & \vdots & \ddots & \vdots \\
a_{n1} & a_{n2} & \dots & a_{nm}
\end{bmatrix}
\]

\end{exercise}

\begin{tcolorbox}[enhanced jigsaw, left=2mm, colbacktitle=quarto-callout-tip-color!10!white, toprule=.15mm, coltitle=black, leftrule=.75mm, arc=.35mm, colback=white, opacitybacktitle=0.6, rightrule=.15mm, colframe=quarto-callout-tip-color-frame, title=\textcolor{quarto-callout-tip-color}{\faLightbulb}\hspace{0.5em}{Solución}, bottomrule=.15mm, breakable, opacityback=0, bottomtitle=1mm, toptitle=1mm, titlerule=0mm]

\begin{Shaded}
\begin{Highlighting}[]
\CommentTok{\% PREÁMBULO}
\BuiltInTok{\textbackslash{}usepackage}\NormalTok{\{}\ExtensionTok{amsmath}\NormalTok{\}}
\CommentTok{\% CUERPO}
\KeywordTok{\textbackslash{}begin}\NormalTok{\{}\ExtensionTok{document}\NormalTok{\}}
\SpecialStringTok{$$}
\KeywordTok{\textbackslash{}begin}\NormalTok{\{}\ExtensionTok{bmatrix}\NormalTok{\}}
\SpecialStringTok{a\_\{11\} \& a\_\{12\} \& }\SpecialCharTok{\textbackslash{}dots}\SpecialStringTok{ \& a\_\{1m\} }\SpecialCharTok{\textbackslash{}newline}
\SpecialStringTok{a\_\{21\} \& a\_\{22\} \& }\SpecialCharTok{\textbackslash{}dots}\SpecialStringTok{ \& a\_\{2m\} }\SpecialCharTok{\textbackslash{}newline}
\SpecialCharTok{\textbackslash{}vdots}\SpecialStringTok{ \& }\SpecialCharTok{\textbackslash{}vdots}\SpecialStringTok{ \& }\SpecialCharTok{\textbackslash{}ddots}\SpecialStringTok{ \& }\SpecialCharTok{\textbackslash{}vdots}\SpecialStringTok{ }\SpecialCharTok{\textbackslash{}newline}
\SpecialStringTok{a\_\{n1\} \& a\_\{n2\} \& }\SpecialCharTok{\textbackslash{}dots}\SpecialStringTok{ \& a\_\{nm\}}
\KeywordTok{\textbackslash{}end}\NormalTok{\{}\ExtensionTok{bmatrix}\NormalTok{\}}
\SpecialStringTok{$$}
\KeywordTok{\textbackslash{}end}\NormalTok{\{}\ExtensionTok{document}\NormalTok{\}}
\end{Highlighting}
\end{Shaded}

\end{tcolorbox}



\end{document}
